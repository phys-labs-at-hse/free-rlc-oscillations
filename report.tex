\documentclass[a4paper, 12pt]{article}

\usepackage[no-math]{fontspec-xetex}
\setmainfont{CMU Serif}
\usepackage[english, russian]{babel}

\usepackage{blindtext}
\usepackage{microtype}
\usepackage{geometry}

\usepackage{amsmath, amsfonts, amssymb, amsthm, mathtools}
\usepackage{MnSymbol}
\usepackage{physics}

\usepackage{graphicx, wrapfig, caption, subcaption}
\usepackage{color, xcolor}

\usepackage{hyperref}

\graphicspath{{figures/}}
\geometry{margin=6em}
\geometry{bottom=6em}

\hypersetup{colorlinks=true, linkcolor=blue, urlcolor=blue}

\title{Затухающие колебания в RLC контуре}
\author{Николай Грузинов}
\date{\today}

\begin{document}
\maketitle

Поскольку работа предполагала следование методичке, буду краток.
\begin{center}
\includegraphics[width=0.7\linewidth]{curcuit.png}
\end{center}

\section{Задачи}
\begin{enumerate}
\item измерить периоды колебаний при нулевом сопротивлении реостата, проверить формулу $T = 2 \pi \sqrt{LC}$
\item выбрав конкретный конденсатор, подобрать с помощью реостата критическое сопротивление для данного контура
\item для этого конденсатора измерить добротность колебаний для разных сопротивлений с помощью амплитуд на пиках затухающей синусоиды
\item то же самое, только в XY-режиме осциллографа.
\item измерив сопротивление и индуктивность катушки, прикинуть добротность контура по формуле $Q = \frac{1}{R} \sqrt{\frac{L}{C}}$
\item сравнить значения добротности, полученные разными способами, проанализировать возможные расхождения
\end{enumerate}

\section{Результаты}
Значения емкостей конденсаторов измерены в методичке правильно, я проверил.
Катушку использовал на 400 витков.

\begin{tabular}{|c|c|}        \hline
ёмкость, нФ & период, мкс  \\ \hline
21.86       & 77           \\ \hline
33.23       & 92           \\ \hline
50.72       & 110          \\ \hline
70.83       & 130          \\ \hline
100.3       & 160          \\ \hline
223.6       & 240          \\ \hline
477.5       & 340          \\ \hline
927.7       & 480          \\ \hline
\end{tabular}

\begin{center}
\includegraphics[width=0.5\linewidth]{check_periods.png}
\end{center}

Для дальнейших измерений выбрал пятый конденсатор, потому что для конденсаторов с меньшей емкостью нельзя было подобрать критическое сопротивление (сопротивления реостата не хватало), а поскольку в дальнейшем нужны были доли критического сопротивления, то хотелось, чтобы оно составляло большую часть от сопротивления реостата, и конденсаторы с большей емкостью не пододили.

Далее $C =$~100.3~нФ.
Характеристическая частота для 400-витковой катушки и этого конденсатора: $\frac{1}{T} =$~6.4~кГц.
Индуктивность и сопротивление катушки при маленьких частотах: 5.841~мГн, 3.5~Ом.
На частоте 10 кГц: 6.139~мГн, 8.3 Ом.
На характеристической частоте: 5.97~мГн, 6.7~мГн.

Критическое сопротивление для данного контура: $R_{\text{cr}} = 440 \pm 10$~Ом при полносм сопротивлении реостата 466 Ом.
Погрешность оценил по чувствительности пика на картинке осциллографа к изменениям --- при меньших изменениях не видел разницы.

\section{Выводы}

\end{document}
